\documentclass{article}
\usepackage[utf8]{inputenc}
\usepackage[spanish,mexico]{babel}
\usepackage[margin=1.5cm]{geometry}
\usepackage{graphicx}
\usepackage[export]{adjustbox}
\usepackage{caption}
\usepackage{subcaption}
\usepackage{fancyhdr}
\pagestyle{fancy}
\fancyhf{}
\usepackage{booktabs}
\pagestyle{empty}
\usepackage{amsmath, amsthm, amssymb}
\usepackage{tikz}
\usepackage{longtable}
\usepackage{tikz}
\setlength{\parindent}{0px} 
\usepackage{multicol}
\usepackage[usenames]{color}

\title{Ejercicio167}
\author{Uriel Solis Procopio}
\date{\today}

\begin{document}

\thispagestyle{empty}
	
	\includegraphics[width=0.20\textwidth]{Logo IPN.png}
	\includegraphics[width=0.10\textwidth, right=12cm]{ESFM.png}
	
\begin{center}
        	
\LARGE
	Formulario  Simulación II
	
        	
\vspace{0.5cm}
\normalsize	
Solis Procopio Uriel 

\vspace{0.5cm}
\end{center}

\setcounter{page}{1}
\pagestyle{plain}

    \begin{multicols}{2}
    
    \section{Funciones Continuas}   

        \subsection{\textcolor{blue}{Uniforme}}

            Si $X$ es una variable aleatoria continua con distribución uniforme continua entonces escribimos $X \thicksim U(a,b)$

            \subsubsection{Función de densidad}

                Si $X\thicksim U(a,b)$ entonces la función de densidad es

                    \begin{equation*}
                        f_X (x) = \frac{1}{b-a}
                    \end{equation*}

                Para $x\in [a,b]$

            \subsubsection{Función de distribución}

                Si $X\thicksim U(a,b)$ entonces la función de distribución es

                    \begin{equation*}
                        F_X (x)= \left\{ \begin{array}{lcc}
                                 0 &   si  & x < a \\
                                 \\ \frac{x-a}{b-a} &  si & a \leq  x < b \\
                                 \\ 1 &  si  & x \geq b
                                 \end{array}
                                \right.
                    \end{equation*}
                    
            \subsubsection{Media}

                \begin{equation*}
                    E[X] = \frac{a+b}{2}
                \end{equation*}
                
            \subsubsection{Varianza}

                \begin{equation*}
                    \sigma ^2 = \frac{(b-a)^2}{12}
                \end{equation*}

        \subsection{\textcolor{blue}{Exponencial}}

            \subsubsection{Función de densidad}

                Se dice que una variable aleatoria continua $X$ tiene una distribución exponencial con parámetros $\lambda > 0 $ y escribimos $X\thicksim Exp(\lambda)$ con función de densidad

                    \begin{equation*}
                        f_X (x) = \lambda e^{-\lambda x}
                    \end{equation*}

                Para $x>0$

            \subsubsection{Función de distribución}

                La función de distribución esta dada por

                    \begin{equation*}
                        F_X (x) =1-e^{-\lambda x}
                    \end{equation*}

                Para $x\geq 0$ 

            \subsubsection{Media}

                \begin{equation*}
                    E[X] = \frac{1}{\lambda}
                \end{equation*}
                
            \subsubsection{Varianza}

                \begin{equation*}
                    \sigma ^2 = \frac{1}{\lambda ^2}
                \end{equation*}
            

        \subsection{\textcolor{blue}{Weibull}}

            \subsubsection{Función de densidad}

                Si $X$ es una variable aleatoria continua, se dice que $X$ tiene una distribución Weibull con parámetros $\alpha, \lambda >0$ y escribimos $X\thicksim Weibull(\alpha, \lambda)$ y su función de densidad y su función esta dada por

                    \begin{equation*}
                        f(x)= \lambda \alpha (\lambda x)^{\alpha -1} e^{-(\lambda x)^\alpha}
                    \end{equation*}

            \subsubsection{Función de distribución}

                La función de distribución esta dada por

                    \begin{equation*}
                        F(x) = 1-e^{-(\lambda x)^{\alpha}}
                    \end{equation*}

                Para $x>0$

            \subsubsection{Media}

                \begin{equation*}
                    E[X] = \frac{1}{\lambda} \Gamma \left( 1+\frac{1}{\alpha} \right)
                \end{equation*}
                
            \subsubsection{Varianza}

                \begin{equation*}
                    \sigma ^2 = \frac{1}{\lambda ^2} \left[ \Gamma \left( 1+\frac{2}{\alpha} \right) -\Gamma ^2 \left( 1+\frac{1}{\alpha} \right) \right]
                \end{equation*}

                
        \subsection{\textcolor{blue}{Log-Normal}}

            \subsubsection{Función de densidad}

                Una variable aleatoria positiva $X$ tiene una distribución lognormal con parámetros $\mu$  y $\sigma$ y escribimos $X\thicksim Lognormal (\mu, \sigma ^2)$, si el logaritmo natural de $X$ sigue una distribución normal con media $\mu$  y varianza $ \sigma ^2$ esto es  $X\thicksim N (\mu, \sigma ^2)$ para obtener

                    \begin{equation*}
                        f_X (x) = \frac{1}{\sigma x\sqrt{2\pi}e^{-\frac{(\ln x-\mu)^2}{2\sigma ^2}}}
                    \end{equation*}

            \subsubsection{Función de distribución}

                La función de distribución esta dada por

                    \begin{equation*}
                        F_X(x) = \Phi \left( \frac{\ln x -\mu}{\sigma} \right)
                    \end{equation*}

                Donde $\Phi$ es la función de distribución acumulada de una normal estándar $N(0,1)$

            \subsubsection{Media}

                \begin{equation*}
                    E[X] = e^{\mu + \frac{\sigma ^2}{2}}
                \end{equation*}
                
            \subsubsection{Varianza}

                \begin{equation*}
                    \sigma ^2 = \left( e^{\sigma ^2} -1\right) e^{2\mu +\sigma ^2}
                \end{equation*}

        \subsection{\textcolor{blue}{Normal}}

            \subsubsection{Función de densidad}

                Denotaremos a una distribución normal como sigue $X\thicksim N (\mu, \sigma ^2)$ entonces su función de densidad esta dada por
    
                    \begin{equation*}
                        \varphi_{\mu, \sigma ^2} (x) = \frac{1}{\sigma \sqrt{2\pi}} e^{-\frac{(x-\mu)^2}{2\sigma ^2}}, ~~x \in \mathbb{R}
                    \end{equation*}

            \subsubsection{Función de distribución}

                La función de distribución esta dada por

                    \begin{equation*}
                        \Phi _{\mu,\sigma ^2}(x) = \frac{1}{2} \left[ 1+ erf\left( \frac{x-\mu}{\sigma \sqrt{2}} \right) \right], ~~x\in \mathbb{R}
                    \end{equation*}

                Donde {\bf erf = función error de Gauss} que la podemos expresar como

                    \begin{equation*}
                        erf(z) = \frac{2}{\sqrt{\pi}} \int _0^z e^{-t^2} dt
                    \end{equation*}

            \subsubsection{Media}

                \begin{equation*}
                    E[X] = \mu
                \end{equation*}
                
            \subsubsection{Varianza}

                \begin{equation*}
                    \sigma ^2 
                \end{equation*}

    \section{Funciones discretas}

        \subsection{\textcolor{blue}{Bernoulli}}
            
            Si $X$ es una variable aleatoria discreta que mide el número de éxitos y se realiza un único experimento con dos posibles resultados denotamos éxito y fracaso, se dice que la variable aleatoria $X$ se distribuye como una Bernoulli de parámetros $p$ con $0<p<1$ y escribimos   $X\thicksim Bernoulli (p)$
            
            \subsubsection{Función de Probabilidad} 

                Su función de probabilidad es

                    \begin{equation*}
                        P[X=x] = p^x(1-p)^{1-x} ,~~x=0,1
                    \end{equation*}

            \subsubsection{Función de distribución}

                La función de distribución acumulada de una variable aleatoria Bernoulli esta dado por

                    \begin{equation*}
                                    F(x)= \left\{ \begin{array}{lcc}
                                     0 &   si  & x < 2 \\
                                     \\ 1-p &  si & 0 \leq  x < 1 \\
                                     \\ 1 &  si  & x \geq 1
                                     \end{array}
                           \right.
                    \end{equation*}

            \subsubsection{Media}

                \begin{equation*}
                    E[X] = p
                \end{equation*}
                
            \subsubsection{Varianza}

                \begin{equation*}
                    \sigma ^2 = p(1-p)
                \end{equation*}
                
                
        \subsection{\textcolor{blue}{Uniforme discreta}}

                Si $X$ es una variable aleatoria discreta cuyo soporte es el conjunto $\{ x_1,x_2,\ldots , x_n \}$ y tiene una distribución uniforme discreta entonces lo escribiremos como $X\thicksim Uniforme (x_1,x_2,\ldots , x_n)$
                
            \subsubsection{Función de Probabilidad}

                La función de probabilidad es

                    \begin{equation*}
                        P[X=x] = \frac{1}{n}
                    \end{equation*}

                Para $x=x_1,x_2,\ldots , x_n$ 

            \subsubsection{Media}

                \begin{equation*}
                    E[X] = \frac{1}{n} \sum _{i=1}^n x_i
                \end{equation*}
                
            \subsubsection{Varianza}

                \begin{equation*}
                    \sigma ^2 = \frac{1}{n} \sum_{i=1}^n (x_i-E[X])^2
                \end{equation*}
            
        \subsection{\textcolor{blue}{Binomial}}

            Si una variable aleatoria discreta $X$ tiene una distribución binomial con parámetros $n\in \mathbb{N}$ y $p$ con $0<p<1$ entonces escribiremos $X\thicksim Bin (n,p)$

                \subsubsection{Función de Probabilidad}

                    Su función de probabilidad esta dada por


                        \begin{equation*}
                            P[X=x]= \binom{n}{x} p^x (1-p)^{n-x}
                        \end{equation*}

                    Para $x=0,1,2,\ldots , n$, donde 

                        \begin{equation*}
                            \binom{n}{x} = \frac{n!}{x!(n-x)!}
                        \end{equation*}
                        
                \subsubsection{Función de distribución }

                        \begin{equation*}
                            F_X(x) = P[X\leq x]= \sum _{k=0}^x \binom{n}{k} p^k(1-p)^{n-k}     
                        \end{equation*}

                \subsubsection{Media}

                \begin{equation*}
                    E[X] = np
                \end{equation*}
                
                \subsubsection{Varianza}

                \begin{equation*}
                    \sigma ^2 = np(1-p)
                \end{equation*}
                        
        \subsection{\textcolor{blue}{Geométrica}}

            Si una variable aleatoria discreta $X$ sigue una distribución geométrica con parámetros $0<p<1$ entonces escribiremos $X\thicksim Geometrica (p)$ 

                 \subsubsection{Función de Probabilidad}

                    La función de probabilidad podemos verla como

                        \begin{equation*}
                            P[X=x]=p(1-p)^{x-1}
                        \end{equation*}

                    Para $x=1,2,3,\ldots$

                \subsubsection{Función de distribución }

                    La función de distribución esta dada por

                        \begin{equation*}
                            P[X\leq x]  = 1-(1-p)^x
                        \end{equation*}

                    Para $x=0,1,2,3,\ldots $

                \subsubsection{Media}

                            \begin{equation*}
                                E[X] = \frac{1}{p}
                            \end{equation*}
                
                \subsubsection{Varianza}

                    \begin{equation*}
                        \sigma ^2 = \frac{1-p}{p^2}
                    \end{equation*}
                    
        \subsection{\textcolor{blue}{Poisson}}

            Sea $\lambda > 0$ y $X$ una variable aleatoria discreta, si la variable aleatoria $X$ tiene una distribución de Poisson con parámetros $\lambda$ entonces escribiremos $X\thicksim Poisson (\lambda)$ 

                \subsubsection{Función de Probabilidad}

                    La función de probabilidad esta dada por

                        \begin{equation*}
                            P[X=x]=\frac{e^{-\lambda}\lambda ^k}{k!}
                        \end{equation*}

                    Donde $k=0,1,2,3,\ldots$

                \subsubsection{Función de distribución }

                    La función de distribución queda de la siguiente manera

                        \begin{equation*}
                            \frac{\Gamma (\lfloor k+1 \rfloor ,\lambda )}{ \lfloor k \rfloor!}
                        \end{equation*}

                    Para $k \geq 0$ donde $\Gamma (x,y)$ es una función gamma incompleta, es decir

                        \begin{equation*}
                            \Gamma (a,x) = \int _x^{\infty} t^{a-1}e^{-t} dt
                        \end{equation*}

                \subsubsection{Media}

                            \begin{equation*}
                                E[X] = \lambda
                            \end{equation*}
                
                \subsubsection{Varianza}

                    \begin{equation*}
                        \sigma ^2 = \lambda
                    \end{equation*}

        \section{Metodología de la simulación}

            \begin{enumerate}
                \item Identificación de variables: aquellas cuyo comportamiento define el comportamiento o la evolución global del sistema
                \item  Determine la distribución de probabilidad
                \item Modele las variables aleatorias
                \item  Defina el modelo del sistema y los objetivos de la simulación
                \item  Diseño del experimento 
                \item  Repita el experimento $n$ veces
                \item  Obtener la gráfica de estabilización
                \item  Calculo de probabilidad
                \item  Hallar el intervalo de confianza
            \end{enumerate}
                
    \end{multicols}

\end{document}
