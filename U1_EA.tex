\documentclass{article}
\usepackage[utf8]{inputenc}
\usepackage[spanish,mexico]{babel}
\usepackage[margin=1.5cm]{geometry}
\usepackage{graphicx}
\usepackage[export]{adjustbox}
\usepackage{caption}
\usepackage{subcaption}
\usepackage{fancyhdr}
\pagestyle{fancy}
\usepackage{booktabs}
\pagestyle{empty}
\usepackage{amsmath, amsthm, amssymb}
\usepackage{tikz}
\usepackage{longtable}

\title{U3_EA}
\author{Uriel Solis Procopio}
\date{\today}

\begin{document}

\thispagestyle{empty}
	
	\includegraphics[width=0.30\textwidth]{Logo UnADM.png}
	\includegraphics[width=0.33\textwidth, right=12cm]{Logo matematicàs.jpg}
	
	\begin{center}
	\vspace{0.8cm}
	\LARGE
	UNIVERSIDAD ABIERTA Y A DISTANCIA DE MÉXICO 
	
	\vspace{0.8cm}
	\LARGE
	LICENCIATURA EN MATEMÁTICAS 
	
    \vspace{1.1cm}
	\normalsize	
	ALUMNO \\
	\vspace{.3cm}
	\large
	\textbf{URIEL SOLIS PROCOPIO}

	\vspace{1.1cm}
	\normalsize	
	MATRICULA \\
	\vspace{.3cm}
	\large
	\textbf{ES1821017223}
	
	\vspace{1.1cm}
	\normalsize	
	DOCENTE \\
	\vspace{.3cm}
	\large
	\textbf{LUIS FERNANDO OROZCO CORTES}
	
	\vspace{1.1cm}
	\normalsize	
	ASIGNATURA \\
	\vspace{.3cm}
	\large
	\textbf{ECUACIONES DIFERENCIALES PARCIALES}
	
	\vspace{1.1cm}
	\normalsize	
	UNIDAD 1 \\
	\vspace{.3cm}
	\large
	\textbf{EVIDENCIA DE APRENDIZAJE}
	
	\vspace{1.1cm}
	\today
	\end{center}
	
	\newpage
	
	\setcounter{page}{2}
	\pagestyle{plain}
		
		\begin{enumerate}
			\item $\varphi _x -\varphi _y +(x+y)\varphi =0$
				
				{\bf\underline{Solución:}}\\
					
					La EDP podemos verla de la siguiente manera
						
						\begin{equation*}
							\varphi _x- \varphi _y = -(x+y)\varphi 
						\end{equation*}
					
					Establecemos el siguiente sistema de ecuaciones
						
						\begin{align}
							&\frac{dx}{dt} =1\\
							&\frac{dy}{dt} =-1\\
							&\frac{d\varphi}{dt} = -(x+y)\varphi
						\end{align}						
					
					Aplicando la regla de la cadena para obtener
						
						\begin{equation*}
							\frac{\frac{dy}{dt}}{\frac{dx}{dt}} = \frac{dy}{dx} = \frac{-1}{1} \Longrightarrow \frac{dy}{dx} =-1
						\end{equation*}											
					
					Resolvemos la EDO por el método de variables separables
						
						\begin{equation*}
							\frac{dy}{dx} =-1 \Longrightarrow dy=-dx
						\end{equation*}
					
					Integrando 
						
						\begin{equation}
							\int dy =-\int dx \Longrightarrow y=-x+cte_1 \Longrightarrow cte_1 =y+x
						\end{equation}
					
					Las cuales son las curvas características.\\
					
					Resolvemos (3) por el  método de separación de variables
						
						\begin{equation*}
							\frac{d\varphi}{dt} = -(x+y)\varphi \Longrightarrow \frac{d\varphi}{\varphi} =-(x+y)dt 
						\end{equation*}
					
					Integrando
						
						\begin{equation}
							\int \frac{d\varphi}{\varphi} =-(x+y) \int dt \Longrightarrow \ln (\varphi) = -(x+y)\cdot t+ cte_2
						\end{equation}
					
					Resolvemos (1) por el método de separación de variables 
						
						\begin{equation*}
							\frac{dx}{dt} =1 \Longrightarrow dx=dt 
						\end{equation*}						 
					
					Integrando
						
						\begin{equation*}
							\int dx =\int dt \Longrightarrow x=t+cte_3 \Longrightarrow t=x-cte_3
						\end{equation*}
					
					Sustituimos el valor de $t$ en (5) para obtener
						
						\begin{align*}
							&\ln (\varphi) =(x+y)(x-cte_3)+cte_2\\
							&\ln (\varphi) =-(x^2-xcte_3+xy-ycte_3)+cte_2\\
							&\ln (\varphi) =-x^2+xcte_3-xy+ycte_3+cte_2\\
							&\ln (\varphi) = -x^2-xy+(x+y+1)cte_4
						\end{align*}						 
					
					La $cte_4$ es constante a lo argo de cada curva característica, es decir, varia de cada curva a otra, si tomamos una 						sola curva característica entonces $cte_4$ es constante en toda esa curva tomada, en este sentido si cambiamos de 							curva característica el valor de $cte_4$ cambiará en función de la curva carcaterística seleccionada, así podemos 							decir que $cte_4$ es una función que depende de las curvas características, de esta forma obtendremo
						
						\begin{align*}
							&\ln (\varphi) = -x^2-xy+(x+y+1)f(cte_1)\\
							&\ln (\varphi) = -x^2-xy +(x+y+1)f(x+y)
						\end{align*}
						
					
					Aplicamos la exponencial
						
						\begin{align*}
							&e^{\ln (\varphi)} = e^{-x^2-xy +(x+y+1)f(x+y)}\\
							&\varphi = e^{-x^2-xy +(x+y+1)f(x+y)}
						\end{align*}
					
					{\bf\underline{Comprbación:}}
						
						Calculamos la derivadas parciales
							
							\begin{align*}
								&\frac{\partial \varphi}{dx} = e^{-x^2-xy +(x+y+1)f(x+y)} \cdot (-2x-y+(x+y+1)f'(x+y)+f(x+y))\\
								&\frac{\partial \varphi}{dy} = e^{-x^2-xy +(x+y+1)f(x+y)} \cdot (-x+(x+y+1)f'(x+y)+f(x+y))
							\end{align*}
						
						Sustituyendo las derivadas parciales y la funcón $\varphi$ en la EDP principal para obtener
							
							\begin{align*}
								&\varphi _x -\varphi _y +(x+y)\varphi =0\\
								&e^{-x^2-xy +(x+y+1)f(x+y)} \cdot (-2x-y+(x+y+1)f'(x+y)+f(x+y)) - \\
								&\left( e^{-x^2-xy +(x+y+1)f(x+y)} 									\cdot (-x+(x+y+1)f'(x+y)+f(x+y)) \right) +(x+y)(e^{-x^2-xy +(x+y+1)f(x+y)})=0
							\end{align*}
						
						Factorizamos como termino común a $e^{-x^2-xy +(x+y+1)f(x+y)}$, entonces
							
							\begin{align*}
								&e^{-x^2-xy +(x+y+1)f(x+y)} \left[ -2x-y+(x+y+1)f'(x+y)+f(x+y) +x-(x+y+1)f'(x+y)-f(x+y)+x+y \right]=0\\
								&e^{-x^2-xy +(x+y+1)f(x+y)} [0] =0\\
								&0=0
							\end{align*}
						
						Por tanto la solución a la que llegamos es valida.
						
					
					
			\item $(x-y)\Psi =xy(\Psi _x -\Psi _y)$
				
				{\bf\underline{Solución:}}\\
				
					La EDP podemos verla como
						
						\begin{equation*}
							xy\Psi _x -xy\Psi _y =(x-y)\Psi 
						\end{equation*}
					
					Establecemos el siguiente sistema de ecuaciones
						
						\begin{align}
							&\frac{dx}{dt} = xy\\
							&\frac{dy}{dt} =-xy\\
							&\frac{d \Psi}{dt} = (x-y)\Psi
						\end{align}
					
					Aplicamos la regla de la cadena para obtener
						
						\begin{equation*}
							\frac{\frac{dy}{dt}}{\frac{dx}{dt}} = \frac{dy}{dx} = \frac{-xy}									{xy} \Longrightarrow \frac{dy}{dx} =-1
						\end{equation*}
					
					Resolviendo la EDO por el método de variables separables 
						
						\begin{equation*}
							\frac{dy}{dx} = -1 \Longrightarrow dy=-dx
						\end{equation*}
					
					Integrando obtendremos
						
						\begin{equation*}
							\int dy= -\int dx \Longrightarrow y=-x+cte_1 \Longrightarrow 									cte_1=y+x
						\end{equation*}
					
					Estas son las curvas características.\\
					
					Resolvemos a (8) por el métodode de separación de variables
						
						\begin{equation*}
							\frac{d\Psi}{dt} =(x-y)\Psi \Longrightarrow \frac{d\Psi}{\Psi} 									=(x-y)dt
						\end{equation*}
					
					Integrando 
						
						\begin{equation}
							\int \frac{d\Psi}{\Psi} = (x-y)\int dt \Longrightarrow \ln (\Psi) 								= (x-y)t+cte_2
						\end{equation}
					
					Resolvemos a (6) por el método de separación de variables 
						
						\begin{equation*}
							\frac{dx}{dt} =xy \Longrightarrow \frac{dx}{x}=ydt
						\end{equation*}
					
					Integrando 
						
						\begin{equation*}
							\int \frac{dx}{x} =y\int dt \Longrightarrow \ln (x)=yt+cte_3 									\Longrightarrow \frac{\ln (x)-cte_3}{y} =t
						\end{equation*}
					
					Sustituimos el valor de $t$ en (9) para obtener
						
						\begin{align*}
							&\ln (\Psi)=(x-y)\left( \frac{\ln (x)-cte_3}{y} \right)+cte_2\\
							&\ln (\Psi)= \frac{x\ln (x)-xcte_3-y\ln(x)+ycte_3}{y} +cte_2\\
							&\ln (\Psi)=\frac{x}{y}\ln (x)-\ln (x)+cte_4\left(1-\frac{x}{y}									\right)\\
							&\ln (\Psi)=\frac{x}{y}\ln (x)-\ln (x)+f(x+y)\left(1-\frac{x}{y}									\right)\\ 
							&e^{\ln (\Psi)} = e^{\frac{x}{y}\ln (x)-\ln (x)+f(x+y)\left(1-									\frac{x}{y}\right)}\\
							&\Psi=e^{\frac{x}{y}\ln (x)-\ln (x)+f(x+y)\left(1-\frac{x}{y}									\right)}
						\end{align*}		
					
					
			\item $yu_y +(y+u)u_x=x-y$
				
				{\bf\underline{Solución:}}\\
				
					
					Establecemos el siguiente sistema de ecuaciones
						
						\begin{align}
							&\frac{dx}{dt}=y+u\\
							&\frac{dy}{dt}=y\\
							&\frac{du}{dt}=x-y
						\end{align}
					
					Apliamos la regla de la cadena para obtener
						
						\begin{equation*}
							\frac{\frac{dy}{dt}}{\frac{dx}{dt}} = \frac{dy}{dx} = \frac{y}{y+u} \Longrightarrow \frac{dy}{dx}=\frac{y}{y+u}
						\end{equation*}
					
					Resolvemos la EDO por el método de variables separables
						
						\begin{equation*}
							(y+u)dy=ydx \Longrightarrow ydy+udy=ydx \Longrightarrow dy+\frac{u}{y} dy = dx
						\end{equation*}
					
					Integrando obtendremos
						
						\begin{equation}
							\int dy + u \int \frac{1}{y} dy=\int dx \Longrightarrow y+u\ln (y) =x+ cte_1 \Longrightarrow cte_1=y+u\ln (									y)-x
						\end{equation}
					
					La cuales son las curvas características.\\
					
					Resolvemos a (12) el método de varaibles separables
						
						\begin{equation*}
							\frac{du}{dt} =x-y \Longrightarrow du=(x-y)dt
						\end{equation*}
					
					Integrando
						
						\begin{equation}
							\int du = (x-y)\int dt \Longrightarrow u=(x-y)t+cte_2
						\end{equation}
					
					Resolvemos a (11) 
						
						\begin{equation*}
							\frac{dy}{dt} =y \Longrightarrow  \frac{1}{y} dy =dt
						\end{equation*}
					
					Integrando
						
						\begin{equation}
							\int \frac{1}{y} dy = \int dt \Longrightarrow \ln (y)+cte_3 =t
						\end{equation}
					
					Sustituyendo el valor de $t$ en (14) para obtener
						
						\begin{align*}
							u&=(x-y)(\ln (y)+cte_3)+cte_2\\
							u&=x\ln (y)+xcte_3-ycte_3 -y\ln (y) +cte_2\\
							u&=(x-y)(\ln(y)) + cte_4(x-y-1)\\
							u&=(x-y)(\ln(y)) + f(y-x+u\ln (y))(x-y-1)
						\end{align*}
					
					\section*{Referencias:}
			
			\begin{itemize}
				\item Universidad Abierta y a Distancia de México. Ecuaciones diferenciales parciales. Unidad 1.Preliminares.Ciudad de México, México.
			\end{itemize}
					
		\end{enumerate}
\end{document}